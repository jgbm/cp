\documentclass{article}

\usepackage{amsmath,infer,mathwidth,stmaryrd}

\newcommand{\cut}[4]{\mathbf{cut}\: [#1:#2] \: (#3 \mid #4)}
\newcommand{\unroll}[2]{\mathbf{rec}\:#1.#2}
\newcommand{\roll}[5]{\mathbf{corec}\:#1 [#2:#3] (#4,#5)}
\newcommand{\case}[2]{\mathbf{case}\:#1\:\{#2\}}
\newcommand{\sel}[3]{#1/\mathrm{#2}.#3}

\newcommand{\fragment}[2]{\langle #1 \rangle \, #2}

\def\tensor{\otimes}
\def\parr{\bindnasrepma}
\def\with{\binampersand}
\def\link{\leftrightarrow}
\def\goesto{\Longmapsto}
\def\F{\mathcal{F}}

\def\Fr{\mathbf{Fr}}
\def\Unfr{\mathbf{Unfr}}

\newenvironment{syntax}%
{\[\begin{array}{@{}lr@{\hspace{2px}}c@{\hspace{2px}}l}\ignorespaces}
{\end{array}\]\ignorespacesafterend}

\begin{document}

\section{Syntax}

We start with the base CP syntax:
\begin{syntax}
  \text{Types} & A, B & ::= & A \tensor B \mid A \parr B \mid A \oplus B \mid A \with B \\
  & & \mid & 1 \mid \bot \mid 0 \mid \top \\
  & & \mid & X \mid X^\perp \mid \mu G \mid \nu G \mid {!A} \mid {?A} \\
  \text{Operators} & G & ::= & X.A \\
  \text{Names} & x,y \\
  \text{Terms} & P,Q,R & ::= & x[y].(P \mid Q) \mid x(y).P \mid x[].0 \mid x().P \\
  & & \mid & \sel{x}{inl}{P} \mid \sel{x}{inr}{P} \mid \case{x}{P;Q} \mid \case{x}{} \\
  & & \mid & \unroll{x}{P} \mid \roll{x}{y}{A}{P}{Q} \mid {!x(y).P} \mid {?x[y].P} \\
  & & \mid & \cut{x}{A}{P}{Q} \\
  \text{Contexts} & Z & ::= & x_1:A_1 \dots x_n:A_n
\end{syntax}
We have the usual typing judgment: $P \vdash Z$.  We extend this with a notion of
\textit{fragments} of CP expressions, as follows:
\begin{syntax}
 \text{Fragments} & F & ::= & \fragment{Z}{P} \mid x[y].(\F \mid \F') \mid x(y).\F \mid x[].0 \mid x().\F \\
 & & \mid & \sel{x}{inl}{\F} \mid \sel{x}{inr}{\F} \mid \case{x}{\F;\F'} \mid \case{x}{} \\
 & & \mid & \unroll{x}{\F} \mid \roll{x}{y}{A}{\F}{Q} \mid {!x(y).\F} \mid {?x[y].\F} \\
 \text{Fragment sets} & \F & ::= & \{ F_1, \dots, F_n \}
\end{syntax}
and use $\F$ to denote a set of fragments.  We write $\{F_1,F_2,\dots,\F\}$ to denote
$\{F_1,\dots\}\cup\F$.  We extend the typing judgments by
\[
  \fragment{Z}{P} \vdash Z' \iff P \vdash Z,Z',
\]
and if $\F = \{F_1,\dots,F_n\}$, then
\[
  \F \vdash Z_1 \cup \cdots \cup Z_n \iff F_1 \vdash Z_1 \land \cdots \land F_n \vdash Z_n.
\]

\section{Fragment and Defragmentation}

\begin{align*}
  \Fr_Z(\cut{x}{A}{x \link y}{P}) &\mapsto \Fr_Z(P[y/x]) \\
  \Fr_Z(\cut{x}{A}{P}{Q}) &\mapsto \Fr_{x:A,Z}(P) \cup \Fr_{x:A^\perp,Z}(Q) \\
  \Fr_Z(P) &\mapsto \{ \fragment{Z}{P} \}
\end{align*}

\noindent
If \(Z = x_1:A_1, x_2,A_2, \dots, x_n:A_n,\) then \(Z \setminus x_i = x_1:A_1,\dots,x_{i-1}:A_{i-1},x_{i+1}:A_{i+1},\dots,x_n:A_n.\)

\begin{align*}
  \Unfr(\{\fragment{}{P}\}) &\mapsto P \\
  \Unfr(\{\fragment{x:A}{P},\F\}) &\mapsto \cut{x}{A}{P}{\Unfr(\{\fragment{Z \setminus x}{Q} \mid \fragment{Z}{Q} \in \F\})} \\
  \Unfr(\{\fragment{x:{!A},Z}{P},\F\}) &\mapsto \Unfr(\F) \quad \text{if $x:{?A} \notin Z'$ for all $\fragment{Z'}{Q} \in \F$}
\end{align*}

We should have that if $P \vdash Z$ then $\Unfr(\Fr_Z(P)) \equiv P$ and $\Fr_Z(P) \vdash Z$.

\section{Structural equivalence}

If $x \not\in fn(P)$:
\[ \{ \fragment{x:A,Z}{P}, \F \} \goesto \{ \fragment{Z}{P}, \F \}. \]

I am somewhat suspicious of this rule in hindsight.  In particular, I would like a property that
says something like $\F \vdash Z \land \F \goesto \F' \implies \F' \vdash Z$, but this rule violates
that unless $A$ is of the form $?B$.  Alternatively, I may need a more refined notion of fragment
typing that takes free names into account.

\section{Principal cut reductions}

Cut rule:
\[ \cut{x}{A \tensor B}{x[y].(P|Q)}{x(y).R} \goesto \cut{y}{A}{P}{\cut{x}{B}{Q}{R}} \]
Fragment transformation:
\begin{multline*}
 \{\fragment{x:A \tensor B,Z}{x[y].(P|Q)}, \fragment{x:A \parr B,Z'}{x(y).R}, \F \} \goesto \\
 \{\fragment{y:A,Z}{P}, \fragment{y:A^\perp,x:B,Z}{Q}, \fragment{y:A^\perp,x:B^\perp,Z'}{R}, \F \}
\end{multline*}

\noindent
Cut rule:
\[ \cut{x}{A \oplus B}{\sel{x}{inl}{P}}{\case{x}{Q;R}} \goesto \cut{x}{A}{P}{Q} \]
Fragment transformation:
\begin{multline*}
  \{\fragment{x:A \oplus B,Z}{\sel{x}{inl}{P}}, \fragment{x:A \with B,Z'}{\case{x}{Q;R}},\F\} \goesto \\
  \{\fragment{x:A,Z}{P}, \fragment{x:A^\perp,Z'}{Q},\F\}
\end{multline*}

\noindent
Cut rule:
\[ \cut{x}{\exists X.B}{x[A].P}{x(X).Q} \goesto \cut{x}{B[A/X]}{P}{Q[A/X]} \]
Fragment transformation:
\begin{multline*}
  \{\fragment{x:\exists X.B,Z}{x[A].P}, \fragment{x:\forall X.B,Z'}{x(X).P}, \F \} \goesto \\
  \{\fragment{x:B[A/X],Z}{P}, \fragment{x:(B[A/X])^\perp,Z'}{Q[A/X]}, \F\}
\end{multline*}

\noindent
Cut rule:
\[ \cut{x}{1}{x[].0}{x().P} \goesto P \]
Fragment transformation:
\[ \{ \fragment{x:1}{x[].0}, \fragment{x:\bot,Z}{x().P}, \F \} \goesto \{ \fragment{Z}{P}, \F \} \]

\noindent
Cut rule:
\begin{multline*}
 \cut{x}{\mu G}{\unroll{x}{P}}{\roll{x}{y}{S}{Q}{R}} \goesto \\
 \cut{y}{S}{Q}{\cut{x}{\overline{G}S}{R}{\cut{z}{\overline{G}(\nu \overline{G})}{F_B((\roll{z}{y}{S}{y \link x}{R[z/x]})_{xz})}{P[z/x]}}}
\end{multline*}
Fragment transformation:
\begin{multline*}
  \{\fragment{x:\mu G,Z}{\unroll{x}{P}}, \fragment{x:\nu G,Z'}{\roll{x}{y}{S}{Q}{R}}, \F \} \goesto \\
  \{\fragment{y:S,Z'}{Q}, \fragment{y:S^\perp,x:\overline{G}S,Z'}{R},\\
    \fragment{y:S^\perp,x:(\overline{G}S)^\perp,z:\overline{G}(\nu \overline{G}),Z'}{F_B((\roll{z}{y}{S}{y \link x}{R[z/x]})_{xz})}, \\
    \fragment{y:S^\perp,x:(\overline{G}S)^\perp,z:(\overline{G}(\nu \overline{G}))^\perp,Z}{P[z/x]}, \F \}
\end{multline*}

\noindent
Cut rules
\begin{align*}
  \cut{x}{{!A}}{!x(y).P}{?x[y].Q} &\goesto \cut{y}{A}{P}{Q} \\
  \cut{x}{{!A}}{!x(y).P}{Q} &\goesto Q \\
  \cut{x}{{!A}}{!x(y).P}{Q[x/x',x/x'']} &\goesto \cut{x'}{{!A}}{!x'(y).P}{\cut{x''}{{!A}}{!x''(y).P}{Q}}
\end{align*}
Fragment transformation:
\begin{multline*}
  \{\fragment{x:{!A},Z}{!x(y).P}, \fragment{x:{?A},Z'}{?x[y].Q}, \F \} \goesto \\
  \{\fragment{x:{!A},Z}{!x(y).P}, \fragment{y:A,Z}{P}, \fragment{y:A^\perp,Z'}{Q}, \F \}
\end{multline*}
and see $\Unfr$ for weakening.

Again, I would like to have the property that $\F \vdash Z \land \F \goesto \F' \implies \F' \vdash
Z.$ This should follow from the corresponding property for cut reduction.


\section{Commuting conversions}

Commuting conversions are written $\{ \F \} \goesto \pi\,\{\F\}$, where each $\pi$ is the prefix
of a CP expressions.  Each rule assumes the side condition $x \notin Z$.

\begin{align*}
  \{ \fragment{Z}{x(y).P}, \F \} &\goesto x(y).\{\fragment{Z}{P}, \F\} \\
  \{ \fragment{Z}{x[y].(P|Q)}, \F \} &\goesto \{ x[y].(\{\fragment{Z}{P}, \F_1\},\{\fragment{Z}{Q}, \F_2\}),\F_3 \} \\
  & \text{if $\forall \fragment{Z_1}{P_1} \in \F_1,\fragment{Z_2}{P_2},dom(Z_1) \cap dom(Z_2) = \emptyset$} \\
  \{ \fragment{Z}{\sel{x}{inl}{P}}, \F \} &\goesto \sel{x}{inl}{\{\F\}} \\
  \{ \fragment{Z}{\case{x}{P;R}}, \F \} &\goesto \case{x}{\{\fragment{Z}{P},\F\};\{\fragment{Z}{Q},\F\}} \\
  \{ \fragment{Z}{\unroll{x}{P}}, \F \} &\goesto \unroll{x}{\{\fragment{Z}{P}, \F\}} \\
  \{ \fragment{Z}{\roll{x}{y}{S}{P}{Q}}, \F \} &\goesto \roll{x}{y}{S}{\{\fragment{Z}{P}\}}{Q} \\
  \{ \fragment{Z}{x().P}, \F \} &\goesto x().\{\fragment{Z}{P},\F\} \\
  \{ \fragment{Z}{!x(y).P}, \F\} &\goesto !x(y).\{\fragment{Z}{P}, \F\} \\
  &\text{if $\forall \fragment{Z'}{P'} \in \F,\forall x:A \in Z', A = {?B}$} \\
  \{ \fragment{Z}{?x[y].P}, \F\} &\goesto ?x[y].\{\fragment{Z}{P}, \F\}
\end{align*}

Finally, we can extend reduction to take place under prefixes:
\[
  \pi \F \goesto \pi \F' \quad \text{if $\F \goesto \F'$}
\]

\section{Reduction}

If $P \vdash Z$, and applying cut elimination to $P$ produces $Q \vdash Z$, then we should have that
$\Fr_Z(P) \goesto \F$, $\Unfr(\F) = Q'$, and both $Q \equiv Q'$ and $Q' \vdash Z$.

\end{document}
