% -*- fill-column: 100 -*-

\documentclass[orivec,envcountsame]{llncs}

\usepackage{amsmath,amssymb,centernot,hyperref,mathwidth,stmaryrd,url,xspace}
\usepackage[table]{xcolor}
\usepackage{colortbl}
\addtolength{\jot}{-3px}

\usepackage{mathpartir}
\mprset{sep=1em}

\usepackage{float}
\floatstyle{boxed}
\restylefloat{figure}

\DeclareMathOperator{\dom}{dom}

%% \newcommand{\Cx}{\mathcal{C}}
%% \newcommand{\Ex}{\mathcal{E}}
%% \newcommand{\Fx}{\mathcal{F}}

\newcommand{\gvtimes}{\times} %\boldsymbol{\otimes}}
\newcommand{\gvplus}{+} %\boldsymbol{\oplus}}

\newcommand{\Cx}{G}
\newcommand{\Ex}{E}
\newcommand{\Fx}{F}

\newcommand{\one}{\mathbf{1}}
\newcommand{\zero}{\mathbf{0}}
\newcommand{\with}{\mathbin\binampersand}
\newcommand{\parr}{\mathbin\bindnasrepma}
\newcommand{\cpdual}[1]{#1^\perp}
\newcommand{\cpbang}[1]{{! #1}}
\newcommand{\cpquery}[1]{{? #1}}
\newcommand{\gvdual}[1]{\overline{#1}}
\newcommand{\channel}[1]{#1^\sharp}
\newcommand{\gvout}[2]{\ensuremath{{!#1.#2}}}
\newcommand{\gvin}[2]{\ensuremath{{?#1.#2}}}
\newcommand{\gvend}{\mkwd{end}}
\newcommand{\lto}{\ensuremath{\multimap}}
\newcommand{\gvforever}[1]{\ensuremath{\Box #1}}
\newcommand{\outterm}{\mkwd{end}_!}
\newcommand{\interm}{\mkwd{end}_?}
\newcommand{\gvserver}[1]{\flat #1}
\newcommand{\gvservice}[1]{\sharp #1}
\newcommand{\trule}[1]{(\textit{#1})}
\newcommand{\cptyp}[2]{#1 \vdash #2}
\newcommand{\gvtyp}[3]{#1 \vdash #2 : #3}

\newcommand{\defiff}{\stackrel{\mathrm{def}}{\iff}}
\newcommand{\defeq}{\triangleq}

\newcommand{\mapname}{\mathsf{map}}
\newcommand{\map}[3]{\mapname^{#1}_{#2}(#3)}
\newcommand{\gvmap}[3]{\mapname^{#1}_{#2}\:#3}

\newcommand{\teval}{\longrightarrow_V}
\newcommand{\ceval}{\longrightarrow}

\newcommand{\rto}{\longrightarrow}
\newcommand{\crto}{\longrightarrow_C}
\newcommand{\ccto}{\longrightarrow_{CC}}

\newcommand{\bigrto}{\Longrightarrow}
\newcommand{\bigcrto}{\Longrightarrow_C}
\newcommand{\bigccto}{\Longrightarrow_{CC}}

\newcommand{\fv}{\ensuremath{\mathsf{fv}}}

\newcommand{\bind}[2]{\ensuremath{#2 / #1}}

\newcommand{\mkwd}[1]{\mathsf{#1}}
\newcommand{\tkwd}[1]{\textsf{#1}}
\newcommand{\link}[2]{#1 \leftrightarrow #2}
\newcommand{\cut}[4]{\nu#1 \: (#3 \mid #4)}
\newcommand{\replicate}[2]{{!#1(#2)}}
\newcommand{\derelict}[2]{{?#1[#2]}}
\newcommand{\rec}[1]{\mkwd{rec}\:#1}
\newcommand{\corec}[4]{\mkwd{corec}\:{#1} \langle #2 \rangle (#3,#4)}
\newcommand{\corecf}[5]{\mkwd{corec}^{#1} #2 \langle #3 \rangle (#4,#5)}
\newcommand{\clabel}[1]{\mathit{#1}}
\renewcommand{\case}[2]{\mkwd{case}\:#1\:\{#2\}}
\newcommand{\sel}[2]{#1[\clabel{#2}]}
\newcommand{\gvsend}[1]{\gvconst{send} \app #1}
\newcommand{\gvreceive}[1]{\gvconst{receive} \app #1}
\newcommand{\gvlet}[3]{\mkwd{let}\;#1 = #2\;\mkwd{in}\;#3}
\newcommand{\gvletunit}[2]{#1; #2}
\newcommand{\gvconst}[1]{\ensuremath{\mathsf{#1}}}
\newcommand{\app}{\,}
\newcommand{\gvselect}[2]{\mkwd{select}\:#1\:#2}
\newcommand{\gvcase}[2]{\mkwd{case}\:#1\:\{#2\}}
\newcommand{\gvlink}[2]{\gvconst{link}\:(#1, #2)}
\newcommand{\gvfork}[2]{\gvconst{fork}\:(\lambda #1.#2)}
\newcommand{\gvwait}[1]{\gvconst{wait}\:#1}
%% \newcommand{\lrkwd}{\mkwd{cofix}}
%% \newcommand{\gvfix}[3]{\lrkwd\:#1\:#2 = #3}
%% \newcommand{\gvletrec}[4]{\gvlet{#1}{\gvfix{#1}{#2}{#3}}{#4}}
\newcommand{\gvserve}[2]{\gvconst{replicate} \app (\lambda #1.#2)}
\newcommand{\gvrequest}[1]{\gvconst{request} \app #1}

\newcommand{\gvifthen}[3]{\mkwd{if}\:#1\:\mkwd{then}\:#2\:\mkwd{else}\:#3}
\newcommand{\gvreceivek}[4]{\gvlet{({#1}, {#2})}{\gvreceive{#3}}{#4}}

\newcommand{\sd}[1]{\langle#1\rangle}

\newcommand{\clos}[3]{\lambda^{#1}\!#2.#3}

\newcommand{\true}{\mathrm{true}}
\newcommand{\false}{\mathrm{false}}

\newcommand{\key}{\mkwd}
\newcommand{\cofix}{\lrkwd}

\newcommand{\tocpbig}[1]{\left\llbracket #1 \right\rrbracket}
\newcommand{\togvbig}[1]{\stretchleftright{\llparenthesis}{#1}{\rrparenthesis}}

\newcommand{\tocpflip}[1]{\llceil{#1}\rrceil}

\newcommand{\tocp}[1]{\llbracket #1 \rrbracket}
\newcommand{\togv}[1]{\llparenthesis #1 \rrparenthesis}

% CP to GV configuration
%\newcommand{\togvc}[1]{\mathscr{C}\hspace{-.1ex}\togv{#1}}
\newcommand{\togvc}[1]{\mathcal{C}\togv{#1}}

\newcommand{\topi}[1]{({#1})^\star}

\newcommand{\un}{un}
\newcommand{\lin}{lin}

\newcommand{\mucp}{$\mu\mathrm{CP}$\xspace}
\newcommand{\mugv}{$\mu\mathrm{GV}$\xspace}
\newcommand{\gvpi}{$\mu\mathrm{GV}\pi$\xspace}

\newcommand{\ba}{\begin{array}}
\newcommand{\ea}{\end{array}}

\newcommand{\bl}{\ba[t]{@{}l@{}}}
\newcommand{\el}{\ea}

\newenvironment{code}{\small\[\bl}{\el\]\normalsize}

\newenvironment{equations}{\[\ba{@{}r@{~}c@{~}l@{}}}{\ea\]}
\newenvironment{eqs}{\ba{@{}r@{~}c@{~}l@{}}}{\ea}

\newenvironment{syntax}%
{\[\begin{array}{@{}lr@{\hspace{2px}}c@{\hspace{2px}}l}\ignorespaces}
{\end{array}\]\ignorespacesafterend}

\newcommand{\mcl}[2]{\multicolumn{#1}{l}{#2}}

\newcommand{\secref}[1]{(\S\ref{sec:#1})}
\newcommand{\lemref}[1]{Lemma~\ref{thm:#1}}
\newcommand{\thmref}[1]{Theorem~\ref{thm:#1}}
\newcommand{\figref}[1]{Figure~\ref{fig:#1}}

\newcommand{\todo}[1]{{\noindent\small\color{red} \framebox{\parbox{\dimexpr\linewidth-2\fboxsep-2\fboxrule}{\textbf{TODO:} #1}}}}
\newcommand{\nqed}{\ensuremath{\centernot\qed}}


\title{Joke or Something}
\author{J. Garrett Morris \and Sam Lindley \and Philip Wadler}
\institute{The University of Edinburgh \\
           \email{\{Garrett.Morris,Sam.Lindley,Philip.Wadler\}@ed.ac.uk} \vspace{-5mm}}

\begin{document}

\maketitle

\begin{abstract}

\end{abstract}

\section{I'm Pickin' Up Good Vibrations}

I'm intentionally ignoring replication for the time being.  I'm intentionally ignoring CP
polymorphism forever.

\subsection{Syntax}
\newcommand{\one}{\mathbf{1}}
\newcommand{\zero}{\mathbf{0}}
\begin{syntax}
  \text{Session types} & S & ::= & \gvout{T}{S} \mid \gvin{T}{S} \mid % \oplus \{ \overline{l:S} \} \mid \with \{ \overline{l:S} \} \mid
                                   \interm \mid \outterm \mid \channel{S} \\
  \text{Types} & T,U & ::= & S \mid T \times U \mid \one \mid T + U \mid \zero \mid T \lto U \\ % \mid {!T} \\
  \text{Terms} & M,N & ::= & x \mid K \mid \lambda x.M \mid M\,N \\
  & & \mid & (M,N) \mid \gvlet{(x,y)}{M}{N} \\
  & & \mid & \mkwd{inl}\,M \mid \mkwd{inr}\,M \mid \gvcase{M}{\mkwd{inl}\,x \mapsto N; \mkwd{inr}\,x \mapsto N} \\
  & & \mid & () \mid \gvletunit{M}{N} \mid \gvconst{absurd} \app M \\
  \text{Constants} & K & ::= & \gvconst{send} \mid \gvconst{receive} \mid \gvconst{fork} \mid \gvconst{wait} \mid \gvconst{link}\\
\end{syntax}

\begin{mathpar}
\gvdual{\gvout{T}{S}} = \gvin{T}{\gvdual S}

\gvdual{\gvin{T}{S}} = \gvout{T}{\gvdual S}

\gvdual{\interm} = \outterm

\gvdual{\outterm} = \interm
\end{mathpar}

\subsection{Typing}

\begin{mathpar}
\inferrule
  {T \not= \channel{S}}
  {\{x:T\} \vdash x:T}

%% \inferrule
%%   {\Gamma, x:\interm, y:\interm \vdash M: T}
%%   {\Gamma, x:\interm \vdash M\{x/y\}:T}
%%
%% \inferrule
%%   {\Gamma \vdash M: T}
%%   {\Gamma, x:\interm \vdash M:T}
%%
\inferrule
  {\Gamma,x:T \vdash M:U}
  {\Gamma \vdash \lambda x:T.M : T \lto U}

\inferrule
  {\Gamma \vdash M: T \lto U \\
   \Delta \vdash N: T}
  {\Gamma,\Delta \vdash M\,N: U}

\inferrule
  {\Gamma \vdash M: T \\
   \Delta \vdash N: U}
  {\Gamma,\Delta \vdash (M,N): T \times U}

\inferrule
  {\Gamma \vdash M: T \times T' \\
   \Delta,x:T,y:T' \vdash N: U}
  {\Gamma,\Delta \vdash \gvlet{(x,y)}{M}{N} : U}

\inferrule
  {\Gamma \vdash M: T}
  {\Gamma \vdash \mkwd{inl}\,M: T+U}

\inferrule
  {\Gamma \vdash M: T+T' \\
   \Delta,x:T \vdash N: U \\
   \Delta,x:T' \vdash N': U}
  {\Gamma,\Delta \vdash \gvcase{M}{\mkwd{inl}\,x \mapsto N; \mkwd{inr}\,x \mapsto N'} : U}

\inferrule
  { }
  {\vdash (): \one}

\inferrule
  {\Gamma \vdash M: \one \\
   \Delta \vdash N: T}
  {\Gamma,\Delta \vdash \gvletunit{M}{N}: T}

\inferrule
  {\Gamma \vdash M: \zero}
  {\Gamma, \Gamma' \vdash \gvconst{absurd} \app M : T}

\gvconst{send} : T \times \gvout{T}{S} \lto S

\gvconst{receive} : \gvin{T}{S} \lto T \times S

\gvconst{fork} : (S \lto \outterm) \lto \gvdual{S}

\gvconst{wait} : \interm \lto \one

\gvconst{link} : (S, \gvdual{S}) \lto \outterm
\end{mathpar}

\subsection{Evaluation}

\newcommand{\distinguish}[1]{\langle #1 \rangle}

\begin{syntax}
  \text{Values} & V & ::= & x \mid K \mid \lambda x.M \\ % \mid K\,V \\
  & & \mid & () \mid (V,V') \mid \mkwd{inl}\,V \mid \mkwd{inr}\,V  \\
  \text{Expression contexts} & E & ::= &  [] \mid E\,M \mid V\,E \mid \gvletunit{E}{M} \\
  & & \mid & (E,M) \mid (V,E) \mid \gvlet{(x,y)}{E}{M} \\
  & & \mid & \mkwd{inl}\,E \mid \mkwd{inr}\,E \mid \gvcase{E}{\mkwd{inl}\,x \mapsto N; \mkwd{inr}\,x \mapsto N'} \\
  \text{Configurations} & P & ::= & M \mid \distinguish M \mid P \parallel P' \mid (\nu x) P \\
  \text{Configuration contexts} & C & ::= & E \mid \distinguish E \mid C \parallel P \mid (\nu x)C \\
  \text{Flags} & \phi & ::= & \circ \mid \bullet
\end{syntax}
We define combination of flags $\phi + \phi'$ by
\begin{mathpar}
  \circ + \circ = \circ

  \circ + \bullet = \bullet

  \bullet + \circ = \bullet
\end{mathpar}
and leave $\bullet + \bullet$ undefined.  Configuration typing is given by
\begin{mathpar}
\inferrule
  {\Gamma \vdash M: T}
  {\Gamma \vdash^\bullet \distinguish M}

\inferrule
  {\Gamma \vdash M: \outterm}
  {\Gamma \vdash^\circ M}

\inferrule
  {\Gamma, x:\channel{S} \vdash^\phi P}
  {\Gamma \vdash^\phi (\nu x) P}

\inferrule
  {\Gamma,x:S \vdash^{\phi} P \\
   \Gamma',x:\gvdual{S} \vdash^{\phi'} P'}
  {\Gamma,\Gamma',x:\channel{S} \vdash^{\phi + \phi'} P \parallel P'}
\end{mathpar}
We will write $\Gamma \vdash P$ for $\Gamma \vdash^\bullet P$.  Note that while we can write the
classic deadlocked configuration
\[
  (\nu x)(\nu y)(\distinguish{\gvlet{(z,x)}{\gvconst{receive} \app x}{\gvconst{send} \app (z,y)}} \parallel
                 \gvlet{(z,y)}{\gvconst{receive} \app y}{\gvconst {send} \app (z,x)})
\]
it is not well typed: we can only split one of $x,y$ at the $\parallel$.  Term reductions:
\begin{align*}
  (\lambda x:T. M)\,V &\teval M\{V/x\} \\
  \gvletunit{()}{M} &\teval M \\
  \gvlet{(x,y)}{(V,V')}{M} &\teval M\{V/x,V'/y\} \\
  \gvcase{(\mkwd{inl}\,V)}{\mkwd{inl}\,x \mapsto N; \mkwd{inr}\,x \mapsto N'} &\teval N\{V/x\} \\
  E[M] &\teval E[M'] \qquad \text{if $M \teval M'$}
\end{align*}
Configuration (context) equivalence:
\begin{mathpar}
C[\mathsf{link}\,(x,y)] \equiv C[\mathsf{link}\,(y,x)]

C_0 \parallel C_1 \equiv C_1 \parallel C_0

C_0 \parallel (C_1 \parallel C_2) \equiv (C_0 \parallel C_1) \parallel C_2

C_0 \parallel (\nu x)C_1 \equiv (\nu x)(C_0 \parallel C_1) \text{ if $x \not\in \fv(C_0)$}
\end{mathpar}

Awkwardly enough, the equivalence rules do not preserve typing.  For example, supposing that $x \in
\fv(P), y \in \fv(Q), \text{ and } x,y \in \fv(R)$, we have $\vdash (\nu x
y)(P \mid (Q \mid R))$ but not $\nvdash (\nu x y)((P \mid Q) \mid R)$.  We have a simple reduction
rule for configuration types, given by
\begin{mathpar}
\inferrule
%  [T-Output]
  { }
  {\channel{(\gvout T S)} \ceval \channel S}

\inferrule
%  [T-Input]
  { }
  {\channel{(\gvin T S)} \ceval \channel S}

\inferrule
%  [T-Context]
  {T \ceval T'}
  {\Gamma,x:T \ceval \Gamma,x:T'}
\end{mathpar}
Configuration reduction is given by
\begin{mathpar}
\inferrule
  [Send]
  { }
  {C[\gvconst{send} \app (V,x)] \parallel C'[\gvconst{receive} \app x] \ceval C[x] \parallel C'[(V,x)]}

\inferrule
  [New]
  {C \ceval C'}
  {(\nu x)C \ceval (\nu x)C'}

\inferrule
  [Fork]
  {\text{$x$ fresh}}
  {C[\gvconst{fork} \app V] \ceval (\nu x)(C[x] \parallel (V\,x))}

\inferrule
  [Wait]
  { }
  {(\nu x)(C[\gvconst{wait} \app x] \parallel x) \ceval C[()]}

\inferrule
   [Link]
   {x \in \fv(M)}
   {(\nu x) (C[\gvconst{link} \app (x, y)] \parallel C'[M]) \ceval (\nu x)(C[x] \parallel C'[\gvletunit{\gvconst{wait} \app x}{M\{y/x\}}])}

\inferrule
  [Thread]
  {M \teval M'}
  {M \ceval M'}

%% \inferrule
%%    [DLink]
%%    {x \in \fv(M)}
%%    {(\nu x) (C[\mathsf{link}\,(x, y)] \parallel \distinguish M) \ceval (\nu x)(C[x] \parallel \distinguish{\gvletunit{\mathsf{wait}\,x}{M\{y/x\}}})}
%%
\inferrule
  [DThread]
  {M \teval M'}
  {\distinguish M \ceval \distinguish {M'}}

\inferrule
  [Par]
  {C_0 \ceval C_0'}
  {C_0 \parallel C_1 \ceval C_0' \parallel C_1}
%%
%% \inferrule
%%   {C_0 \equiv C_0' \\
%%    C_0' \ceval C_1' \\
%%    C_1' \equiv C_1}
%%   {C_0 \ceval C_1}
\end{mathpar}
Some things that might be useful.
\begin{lemma}\label{thm:term-preservation}
  If $\Gamma \vdash M: T$ and $M \teval M'$, then $\Gamma \vdash M': T$
\end{lemma}

\noindent
The proof is standard.

\begin{theorem}\label{thm:preservation}
  If $\Gamma \vdash P$ and $P \ceval P'$ then there is some $\Gamma'$ such that either $\Gamma
  \ceval \Gamma'$ or $\Gamma = \Gamma'$ and $\Gamma' \vdash P'$.
\end{theorem}

\begin{proof}
  By induction on the structure of $P \ceval P'$.
  \begin{itemize}
  \item Cases \textsc{Thread} and \textsc{DThread} follow from \lemref{term-preservation}, with
    $\Gamma = \Gamma'$.
  \item Cases \textsc{New} and \textsc{Par} are immediate by the induction hypothesis.
  \item In case \textsc{Send}, from the assumption $\Gamma \vdash C[\mkwd{send}\,(V,x)] \parallel
    C'[\gvconst{receive} \app x]$, we can assume that $\Gamma$ partitions as
    $\Gamma_1,\Gamma_2,x:\channel{S}$ such that $S = \gvout{T}{S'}$, $V$ has type $T$, and
    $\Gamma_1,x:\gvout{T}{S'} \vdash C[\gvconst{send} \app (V,x)]$,
    $\Gamma_2,x:\gvin{T}{\gvdual{S'}} \vdash C'[\gvconst{receive} \app x]$.  As $\gvconst{send} : T
    \times \gvout{T}{S'} \lto S'$ and $\Gamma_1,x:\gvout{T}{S'} \vdash C[\gvconst{send} \app
    (V,x)]$, we can conclude that $\Gamma_1,x:S' \vdash C[x]$.  By a similar argument, we conclude
    that $\Gamma_2,x:\gvdual{S'} \vdash C'[(V,x)]$. Finally, as $\channel{(\gvout{T}{S})} \ceval
    \channel{S}$, $\Gamma_1,\Gamma_2,x:\channel{(\gvout T S)} \ceval
    \Gamma_1,\Gamma_2,x:\channel{S}$, and we can recompose the resulting processes concluding that
    $\Gamma_1,\Gamma_2,x:\channel{S} \vdash C[x] \parallel C'[(V,x)]$.
  \item In case \textsc{Fork}, from the assumption $\Gamma \vdash C[\gvconst{fork} \app V]$, we can
    conlude that $\Gamma$ splits into $\Gamma_1,\Gamma_2$ and there is some $S$ such that
    $\Gamma_1,x:S \vdash C[x]$ and $\Gamma_2,x:\gvdual{S} \vdash V \app x$.  Thus we have that
    $\Gamma,x:S \vdash C[x] \parallel V \app x$ and $\Gamma \vdash (\nu x)(C[x] \parallel
    V \app x)$.
  \item In case \textsc{Wait}, from the assumption $\Gamma \vdash (\nu x)(C[\gvconst{wait} \app
    x] \parallel x)$, we can conclude that $\Gamma,x:\interm \vdash C[\gvconst{wait} \app x]$ and
    thus, from the typing of \gvconst{wait}, that $\Gamma \vdash C[()]$.
  \item In case \textsc{Link}, from the assumptions $\Gamma \vdash (\nu x) (C[\gvconst{link} \app
    (x, y)] \parallel C'[M]), x \in \fv(M)$, we can conclude that $\Gamma$ partitions as
    $\Gamma_1,\Gamma_2,y:S$ such that $\Gamma_1,y:S,x:\gvdual{S} \vdash C[\gvconst{link} \app (x,
    y)]$ and $\Gamma_2,x:S \vdash C'[M]$.  (Note that the free variable assumption on the reduction
    rule for \gvconst{fork} allows us to assume that neither $C$ or $C'$ binds $x$ or $y$.)  From
    the type of \gvconst{link}, we have that $\Gamma,x:\outterm \vdash C[x]$; similarly, from $x \in
    \fv(M)$, we can conclude that $\Gamma_2,y:S \vdash C'[M\{y/x\}]$.  Finally, from the typing rule
    for \gvconst{wait}, we have that $\Gamma_2,x:\interm,y:S \vdash C'[\gvbindunit{\gvconst{wait}
      \app x}{M\{y/x\}}]$, and that $\Gamma \vdash (\nu x)(C[x] \parallel
    C'[\gvletunit{\gvconst{wait} \app x}{M\{y/x\}}])$ \qed
  \end{itemize}
\end{proof}

Define $\ceval^\star$ as the reflexive, transitive closure of $\equiv\ceval\equiv$.  The following
is probably not true.
\begin{theorem}\label{thm:normal-forms}
  If $\Gamma \vdash P$, then for all $Q$ such that $P \ceval^\star Q \centernot\ceval$, there is a
  value $V$ such that $Q \equiv (\nu x_1,\dots,x_n)(\distinguish V
  \parallel Q')$.
\end{theorem}
In particular, I have doubts that allowing an arbitrary term to be distinguished is enough to
guarantee that it reduces to a value.

\subsection{Deadlock and Its Absence}

Blocked terms:
\newcommand{\sendp}{\mathsf{send}}
\newcommand{\recvp}{\mathsf{recv}}
\newcommand{\waitp}{\mathsf{wait}}
\newcommand{\blockedp}{\mathsf{blocked}}
\[\begin{array}{rcl}
  \sendp(x,M) &\defiff& \exists N. \; C = \gvconst{send} \app (x,N) \\
  \recvp(x,M) &\defiff& \exists C. \; C = \gvconst{receive} \app x \\
  \waitp(x,M) &\defiff& \sendp(x,M) \lor \recvp(x,M) \\
  \blockedp(x,y,E[M]) &\defiff& \waitp(x,M) \land y \in \fv(E) \\
  \blockedp(x,y,P) &\defiff& (P \equiv C[M] \land \blockedp(x,y,M)) \lor \, (P \equiv C[Q \parallel Q'] \\
  && \quad \land \, (\exists z. \blockedp(x,z,Q) \land \blockedp(z,y,Q')))
\end{array}\]
Finally, we have a natural definition of deadlock:
\newcommand{\deadlockp}{\ensuremath{\mathsf{deadlocked}}}
\[
\deadlockp(P) \defiff P \equiv C[Q \parallel Q'] \land \exists x,y. \blockedp(x,y,Q) \land \blockedp(y,x,Q').
\]
Because this definition depends on blocked processes (that is, those waiting to read or write) it
avoids the problems with return values (since a process returning a channel is not blocked on it, it
cannot be part of a deadlocked pair).

%% And we turn to deadlock.  The ability to return channels makes deadlock freedom tricky to define.
%% Consider the following program
%% \begin{verbatim}
%% let c = fork (\c.
%%           let d = fork (\d.
%%                     let (e, d) = receive d in
%%                     let () = wait e in d) in
%%           let (e, c) = receive c in
%%           let d = send (e, d) in
%%           let () = wait d in c) in c
%% \end{verbatim}
%% The resulting configuration will have a chain of unblockable processes---sending a value on
%% \texttt{c} will unblock the first forked process, which will in turn unblock the second.  Here's a
%% complicated definition of deadlock; hopefully we can simplify.  Let $Q$ be a configuration of the
%% form $(\nu x_1,\dots,x_m)(\distinguish V \parallel E_1 \parallel \dots \parallel E_n)$.  We say that
%% the unblockable channels of $Unb(P)$ is the least set $X$ such that
%% \begin{itemize}
%% \item $\fv(V) \subseteq X$; and,
%% \item if $\mathsf{blocked}(x,E)$ and $x \in X$, then $\fv(E) \subseteq X$.
%% \end{itemize}
%% Now we can say something about deadlock.  A (closed) process $P$ is deadlock-free if, for all $Q$
%% such that $P \ceval^{\star} Q \centernot\ceval$, $Q \equiv (\nu x_1,\dots,x_m)(\distinguish
%% V \parallel E_1 \parallel \dots \parallel E_n)$ and, if $\mathsf{blocked}(x_i,E_j)$ then $x_i \in
%% Unb(Q)$.

\begin{theorem}\label{thm:typed-terms-df}
  If $\Gamma \vdash M: T$, then $\distinguish M$ is deadlock-free.
\end{theorem}

I think at least (and probably at most) one of the following will be necessary.

\begin{lemma}\label{thm:channel-used-twice}
  If $\Gamma \vdash P$ and $P \equiv (\nu x_1,\dots,x_m)(\distinguish{M_1} \parallel M_2 \parallel
  \dots \parallel M_n)$, then for any $1 \leq i \leq m$ there are exactly two indexes $j$, $1 \leq j
  \leq n$, such that $x_i \in \fv(M_j)$.
\end{lemma}

\begin{proof}
  By induction on $n$.  The crux is that each $x_i$ must be split once (as variables of type
  $\channel{S}$ cannot be used in the typing of terms), and that then each instance most be used
  once (by linearity). \qed
\end{proof}

\begin{lemma}\label{thm:procs-share-one-channel}
  If $\Gamma \vdash P$ and $P \equiv (\nu x_1,\dots,x_m)(\distinguish{M_1} \parallel M_2 \parallel
  \dots \parallel M_n)$, then for any pair of terms $M_j,M_{j'}$, there is at most one $x_i$ such
  that $x_i \in \fv(M_j),x_i \in \fv(M_{j'})$.
\end{lemma}

\begin{proof}
  By induction on $n$.
  \begin{itemize}
  \item \textit{Base case:} $n = 2$.  Trivial, following from the typing rule for $\parallel$.
  \item \textit{Inductive case.} Consider the derivation of
    \[ \Gamma,x_1:\channel{S_1},\dots,x_m:\channel{S_m} \vdash Q \parallel M_n. \] Fix $1 \leq i
    \leq m$.  If $x_i$ is not split at this rule, then it cannot appear in $M_n$ (as variables of
    type $\channel{S}$ cannot be used in terms).  Therefore, assume that $x_i$ is split at this
    rule; it must be the only channel appearing in $M_n$.  Finally, note that we have $P \equiv (\nu
    x_i)((\nu x_1,\dots,x_{i-1},x_{i+1},\dots x_m)Q \parallel M)$, where $Q$ has one fewer process
    than $P$, and so the proposition holds for the remaining channels by the induction
    hypothesis. \qed
  \end{itemize}
\end{proof}

I think we can now show fairly directly that

\begin{lemma}\label{thm:typed-configs-df}
  If $\Gamma \vdash P$ then $\lnot \deadlockp(P)$.
\end{lemma}
from which \thmref{typed-terms-df} follows directly.

%% \subsection{Alternative Formulation}
%%
%% Instead, we could have
%%
%% \begin{syntax}
%%   \text{Configurations} & P & ::= & M \mid \distinguish M \mid P \parallel P' \mid (\nu x y)P \\
%%   \text{Configuration contexts} & C & ::= & E \mid \distinguish E \mid C \parallel P \mid (\nu x y)C
%% \end{syntax}
%% with one alternation to the configuration typing rules
%% \[
%% \inferrule
%%   {\Gamma,x:S,y:\gvdual{S} \vdash P}
%%   {\Gamma \vdash (\nu x y)P}
%% \]
%% an additional equivalence
%% \[
%%   (\nu x y)P \equiv (\nu y x)P
%% \]
%% and the following alternations to the reductions
%% \begin{mathpar}
%% \inferrule
%%   { }
%%   {(\nu x y)(C[\mkwd{send}\,(V,x)] \parallel C'[\mkwd{receive}\,y] \parallel C'') \ceval (\nu x y)(C[x] \parallel C'[(V,y)] \parallel C'')}
%%
%% \inferrule
%%   { }
%%   {(\nu x y)(C[\mkwd{wait}\,x] \parallel y \parallel C') \ceval C[()] \parallel C'}
%%
%% \inferrule
%%   {x,y \text{ fresh}}
%%   {C[\mkwd{fork}\,V] \ceval (\nu x y)(C[x] \parallel V\,y)}
%% \end{mathpar}
%% On the one hand, this seems to make the structure of configurations and their reductions slightly
%% more complicated; on the other hand, this avoids either making $\Gamma$ a multi-set or further
%% restricting the form of configurations.  On the whole, I think I marginally prefer this option,
%% although I'm not convinced yet.
%%
%% We would also have to update the definition of blocking.  A process $P$ with free variables
%% $z_1,\dots,z_l$ is deadlock-free if, for all $Q$ such that $P \ceval^{\star} Q \centernot\ceval$, $Q
%% \equiv (\nu x_1 y_1) \dots (\nu x_m y_m)(\distinguish V \parallel E_1 \parallel \dots \parallel E_n)$ and, for
%% each $1 \leq i \leq m, 1 \leq j \leq n$, if $\mathsf{blocked}(x_i,E_j)$ then $y_i \in
%% \fv(V)$ or $x_i \in \{ z_1, \dots, z_l \}$.

\section{CP Bollocks}

Suppose that we write $\tocp{-}$ for GV to CP translation, and $\togv{-}$ for CP to GV translation.
Note that $\tocp{P}z$ will only be defined when $\Gamma \vdash P$, and when $P$ has a single
distinguished expression.  I suspect that the following will be useful at some point.
\begin{lemma}
  If $\Gamma \vdash M: T$ and $\distinguish M \ceval^\star P$, then there are $x_1,\dots,x_n$, $M$ and $P'$
  such that $P \equiv (\nu x_1,\dots,x_n)(\distinguish M \parallel P')$.
\end{lemma}

We might expect results something like the following.

\begin{theorem}
  If $\Gamma \vdash P$, where $P = (\nu x_1,\dots,x_m)(\distinguish M \parallel P')$ and $\Gamma
  \vdash M: T$, then $\tocp{P}z \vdash \tocp{\Gamma}, z:\tocp{T}$.
\end{theorem}

\begin{theorem}
  If $\Gamma \vdash P$ and $P \ceval P'$, then $\tocp{P}z \longrightarrow Q$ such that $\togv{Q}
  \equiv P'$.  What to do with $z$?!
\end{theorem}

\section{Direct GV semantics}

\newcommand{\apar}[1]{\mathbin{{[}\:\!\!\rangle}_{\!{#1}}\!}
\newcommand{\con}[3]{{#1} \apar{#2} {#3}}

\[
\mkwd{connect} : (S \lto \outterm, \gvdual{S} \lto T) \to T
\]

Syntactic sugar:
\begin{align*}
\con{M}{x}{N} &\equiv \mkwd{connect}(\lambda x.M, \lambda x.N) \\
\con{L}{x}{\con{M}{y}{N}} &\equiv \con{L}{x}{(\con{M}{y}{N})} \\
\end{align*}

Equivalences ($\equiv$ is the transitive, reflexive, symmetric closure of the following rules):
\begin{align*}
\mkwd{link}\,(x,y) &\equiv \mkwd{link}\,(y,x) \\
\con{L}{x}{\con{M}{y}{N}} &\equiv \con{M}{y}{\con{L}{x}{N}} \\
\con{L}{x}{\con{M}{y}{N}} &\equiv \con{(\con{L}{x}{M})}{y}{N} \\
\con{L}{x}{\con{M}{y}{N}} &\equiv \con{(\con{M}{x}{L})}{y}{N} \\
\end{align*}

Interesting reductions
\begin{align*}
\con{E[\mkwd{send}\,(V,x)]}{x}{E'[\mkwd{receive}\,x]}
  &\rto \con{E[x]}{x}{E'[(V,x)]} \\
\con{E[\mkwd{receive}\,x]}{x}{E'[\mkwd{send}\,(V,x)]}
  &\rto \con{E[(V,x)]}{x}{E'[x]} \\[1ex]
\con{x}{x}{E[\mkwd{wait}\,x]}
  &\rto E[()] \\
\con{E[\mkwd{wait}\,x]}{x}{x}
  &\rto E[()] \\[1ex]
\con{E[\mkwd{link}\,(x,y)]}{x}{M}
  &\rto \con{E[x]}{x}{(\mkwd{wait}\,x; M[y/x])} \\
\con{M}{x}{E[\mkwd{link}\,(x,y)]}
  &\rto \con{(\mkwd{wait}\,x; M[y/x])}{x}{E[x]} \\[1ex]
\mkwd{link}\,(x,\mkwd{link}\,(y,z))
  &\rto \mkwd{wait}\,x; \mkwd{link}\,(y,z) \\
E[\con{M}{x}{N}]
  &\rto \con{M}{x}{E[N]} \\
\end{align*}

Preservation
\begin{theorem}
If $\Gamma \vdash M : T$ and $M \rto M'$ then there exists $\Gamma'$, such that $\Gamma \rto
\Gamma'$ and $\Gamma' \vdash M' : T$.
\end{theorem}
%
\begin{proof}
By induction on the structure of the derivation $\Gamma \vdash M : T$.
\end{proof}

Progress
\begin{theorem}
If $\Gamma$ is a session environment (all variables in $\Gamma$ have session type), and $\Gamma
\vdash M : T$ then one of the following holds:
\begin{enumerate}
\item $M$ is a value; or
\item there exists $N = \blockedp'(y_1, M_1) \apar{x_1} \dots \apar{x_{n}} \blockedp'(y_{n}, M_{n})$
  such that $M \equiv N$ and $\Gamma \vdash N : T$; or
\item there exists $N$ such that $M \rto N$.
\end{enumerate}
\end{theorem}
%
\begin{proof}
By induction on the structure of $M$.
\end{proof}
% In the second case each $y_i$ must be equal to either $x_i$ or some $x$ in $\Gamma$.

\begin{theorem}
If $\vdash M : T$ then one of the following holds:
\begin{enumerate}
\item $M$ is a value; or
\item there exists $N = \blockedp(x_1, M_1) \apar{x_1} \dots \apar{x_{n-1}} \blockedp(x_{n-1},
  M_{n-1}) \apar{x_n} V$ such that $M \equiv N$ and $\Gamma \vdash N : T$; or
\item there exists $N$ such that $M \rto N$.
\end{enumerate}
\end{theorem}
%
\begin{proof}
By induction on the structure of $M$.
\end{proof}

$\blockedp(x, M)$ means $M$ is blocked on a send or receive involving $x$.

$\blockedp'(x, M)$ means either $M = x$ or $M$ is blocked on a send, receive, link, or wait
involving $x$.

\end{document}
